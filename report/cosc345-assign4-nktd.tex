\documentclass[11pt, oneside]{article}  
\usepackage{geometry}                	
\geometry{a4paper}
\usepackage[parfill]{parskip}		
\usepackage{graphicx}	
\usepackage{pgfgantt}
\usepackage{pgfcalendar}	
\usepackage{pgfkeys}	
\usepackage{amssymb}
\usepackage{lscape}
\usepackage{hyperref}
\hypersetup{
    colorlinks=true,
    linkcolor=blue,
    filecolor=magenta,      
    urlcolor=cyan,
}
\urlstyle{same} 

\title{COSC 345 Assignment 4} \author{Nathan Kennedy \texttt{1006384},
  Kat Lilly \texttt{8867318}, Tobyn Packer \texttt{5003352}, and
  \\Daniel Davidson \texttt{375621}} \date{\today}

\begin{document}

\maketitle
\tableofcontents
\newpage

\section{The NKTD app}

Our app now contains three functioning games which can be opened via
speech commands from the main menu and controlled either by speech
commands or through touch. Two of the games (2048 and Frozen Bubble)
are open source games we have manipulated to work with voice commands
and Teragram is an original game, a maths practice game for children.

The application uses pocketSphinx to interpret speech commands:\linebreak
\href{https://github.com/cmusphinx/pocketsphinx}{https://github.com/cmusphinx/pocketsphinx}

\subsection{Since the Beta}

{\em I have added some new stuff here, but I'm sure I've missed quite a few things -Kat}

We have made the following improvements and enhancements since our
beta release:

\begin{itemize}

\item The whole app now has a consistent design / colour scheme

\item Fixed various bugs - do we want to mention any of these specifically??

\item Fixed frozen bubble launcher thing so that game is now actually playable - launcher stops moving immediately on hearing a command, but waits to interpret it before launching the bubble, this means the delay in interpreting a speech command doesn't ruin the game.

\item Added an animation to the listening button, which is intuitively easy to understand and shows the user when the app has heard something.

\item Changed the way the radio buttons are labelled in multiple choice quizes in Teragram - we originally labelled them as ``a'', ``b'', ``c'' etc, but this was not suitable for voice recognition, so we changed it to numbers, but this was a bit weird looking and a bit distracting. We have changed the labelling of the options to colours instead.

\end{itemize}

\pagebreak

\section{Known Issues and Intended improvements}

{\em Some or all of these issues can be moved to previous section }

\subsubsection*{Speech Recognition off-switch across all games}
We have successfully implemented the on/off switch in the main menu, 2048 and 
Teragram but still have to implement it in FrozenBubble.

\subsubsection*{Visual feedback across all games}
In Teragram, we have successfully implemented visual feedback which
informs the user of how voice inputs are being interpreted, but are
yet to implement it into the main menu, 2048 and FrozenBubble.


\subsubsection*{More robust number entry}
The current way of inputing numbers using speech is finicky and prone to
errors. We are unsure whether or not there is some easier way to do this,
but we would like to try different methods.


\subsubsection*{Frozen Bubble Launcher Speed}
Frozen Bubble is an open source game that is controlled strictly by touch. 
In order for a user to be hands-free, we have had to manipulate the code 
to have the launcher work with respect to time. This currently is working
but not in a functional fashion as it travels far too fast. The issue is 
that the launcher moves with respect to int values, however, for it to be 
functional we need decimals values. When trying to manipulate this it 
throws a chain of bugs where the values are used elsewhere in the game. 


\subsubsection*{Frozen Bubble first level bubbles}
On two of our devices, the initial load of Frozen Bubble has two rows of 
bubbles that are generated directly in-front of the launcher therefore making
the game impossible to play as they are blocking all angles of the launcher. 
Still have debugging to do but currently at a state where all the files have 
been compared with local, working states of the game and no differences are
shown.


\subsubsection*{Help Screen}

We intend to add a help menu on the main page, similar to the help
menus within Teragram and 2048.


\section{Testing}

{\em Do we have anything new to add to this section??}

\subsubsection*{Voice control}

Testing an application that is run by voice commands was not something
we felt could be done through automated methods without devoting a
large amount of time that could instead be spent implementing
features. Instead, frequent testing was done manually either through
emulators or by installing the apk on our own phones. We note that the
microphone on a typical phone is significantly better than the
microphone on a typical laptop or desktop computer, and so the app is
significantly easier to play on a phone than on an emulator.


\subsubsection*{Teragram}

For the Teragram game, we had extensive user testing performed by two
children, seven and four years old. In the months since the alpha and beta
releases, they have still been playing it frequently, so it has passed
the fun test and is clearly suitable for children of those ages, with
some room to grow, as some parts are too hard for them for
now. However, they do prefer to play it without using the voice
control. This is partly because they do not often find themselves in a
quiet enough place for the voice control to work reliably, but also
because they get frustrated that voice controls are a bit slower and
less reliable than entering answers via the touch screen.

\pagebreak

\section{User Documentation}

{\em - I have made some small changes to this section - I think we can leave this mostly unchanged from our last assignment -Kat}

\subsection{Teragram}

The game starts up at level one, and the user can move to the easiest
level by tapping the `too hard' button or go up levels by tapping the
`too easy' button. When ten questions in a row are answered correctly,
the game will automatically level up, otherwise if two questions are incorrect in a 
row, it'll automatically level down.  The user is not aware of the actual level, it's just that
questions get easier or harder - the bound on the random number
generator used in setting new questions is related to the level.

The game launches in addition mode, and the user can chose to move to
subtraction problems, or practice times tables or powers of two.

\subsection{2048}

2048 is a popular number name released on March 20, 2014. This game is
very simple; you can choose to slide the tiles up, down, left or right
by swipping the screen or saying `left', `right' `up' or `down'. Every
slide, all the digital squares will move in the direction the screen was slid. 
This is a really neat game, and there are many open source
implementations out there, but what makes it especially suitable for
our app is the simplicity of its controls - it's an interesting and
challenging game controlled with only four commands. Also there is no
time pressure in this game, so if it takes half a second for the
speech recogniser to work, this does not adversely affect the game
play.

\subsection{Frozen Bubble}
The aim of this game is to clear the screen by shooting the current
bubble from the launcher to its respective coloured bubbles on the
level. Saying `fire'/'now' or swiping up will shoot the bubble.

\subsection{Voice Controls}

You can control the game with the following voice commands:

\subsubsection{Main menu}

\begin{itemize}
  \item {\bf`Teragram'} or {\bf`game two'}  - this will take you to the game Teragram.
  \item {\bf`Twenty Fourty Eight'} or {\bf`game three'} - this will take you to the game 2048.
  \item {\bf`Frozen Bubble'} or {\bf`game four'} - this will take you to the game Frozen Bubble.
\end{itemize}

\subsubsection{Teragram}

\begin{itemize}
  \item {\bf`help'} - Brings up help menu with available voice commands. 
  \item {\bf`harder'} - Provides a harder question. 
  \item {\bf`easier'} - Provides an easier question.
  \item {\bf`new question'} - Provides a new question of the same type
  \item {\bf`addition'} -Provides an addition question
  \item {\bf`subtraction'} - Provides a subtraction question
  \item {\bf`times tables'} - Starts times tables practice mode
  \item {\bf`powers of two'} - Starts powers of two quiz
  \begin{itemize}
  	\item Say the colour of the correct answer: {\bf`blue'}, {\bf`green'}, {\bf`pink'}, or {\bf`red'}
  \end{itemize}
  \item {\bf`number'} - Starts Number Input mode, see below
  \item {\bf`exit'} - Prompts a message confirming if they want to exit
 
\end{itemize}

\subsubsection{2048}

\begin{itemize}
  	\item {\bf`help'} - Brings up help menu with available voice commands. 
	\item {\bf`left'} - All blocks will move to the left if possible. 
	\item {\bf`right'} - All blocks will move to the right if possible
	\item {\bf`up'} - All blocks will move to up if possible. 
	\item {\bf`down'} - All blocks will move to down if possible
	\item {\bf`exit'} - Prompts a message confirming if they want to exit
	
\end{itemize}

\subsubsection{Frozen Bubble}

\begin{itemize}
	\item {\bf`fire'} or {\bf`now'} - Will fire the bubble
	\item {\bf`exit'} - Prompts a message confirming if they want to exit
	
\end{itemize}

\pagebreak

\subsubsection{Number Input}
\begin{itemize}
  \item {\bf numbers 0 to 9 inclusive} – Appends the spoken number
    to the end of your answer, ie. Speaking `one', `two' will give you
    12. Note that 0 is to be spoken `zero'
  \item {\bf `okay'} - Submits the current answer and exits Number
    Input mode
  \item {\bf `back'} - Removes the last number from your answer
  \item {\bf `clear'} - Removes the entirety of your answer
  \item {\bf `cancel'} - Exits Number Input mode
\end{itemize}

When entering numbers, please wait until the number you have just spoken appears
in the box before trying to input the next one.
Good Luck!

\section{Open Source components}

The code for the speech recognizer setup/permissions/asset copying is
taken almost directly from the pocketSphinx demo application here:
\url{https://github.com/cmusphinx/pocketsphinx-android-demo}

The phoneme definitions in our nktd.dic file are selectively taken
from the cmudict-en-us.dict file in that same demo.

The acoustic model we use is provided by CMUSphinx at
\href{https://sourceforge.net/projects/cmusphinx/files/Acoustic\%20and\%20Language\%20Models/US\%20English/}{sourceforge.net}

The game FrozenBubble taken from
\url{https://github.com/kthakore/frozen-bubble}

The game 2048 taken from
\url{https://github.com/BuddyBuild/2048-Android}
    


\end{document}
