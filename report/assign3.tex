\documentclass[11pt, oneside]{article}  
\usepackage{geometry}                	
\geometry{a4paper}
\usepackage[parfill]{parskip}		
\usepackage{graphicx}	
\usepackage{pgfgantt}
\usepackage{pgfcalendar}	
\usepackage{pgfkeys}	
\usepackage{amssymb}
\usepackage{lscape}
\usepackage{hyperref}
\hypersetup{
    colorlinks=true,
    linkcolor=blue,
    filecolor=magenta,      
    urlcolor=cyan,
}
\urlstyle{same} 

\title{COSC 345 Assignment 3} \author{Nathan Kennedy \texttt{1006384},
  Kat Lilly \texttt{8867318}, Tobyn Packer \texttt{5003352}, and
  \\Daniel Davidson \texttt{375621}} \date{\today}

\begin{document}

\maketitle
\tableofcontents
\newpage

\section{Beta Release}

We have now reached the point where our android application consists
of 3 (relatively) functioning games where they are able to be opened
via speech commands from the main menu and controlled either by speech
commands or through touch. Two of the games (2048 and Frozen Bubble)
are open source games we have manipulated to work with voice commands 
and Teragram is an original game.

The application uses pocketSphinx to interpret speech commands
available here:\linebreak
\href{https://github.com/cmusphinx/pocketsphinx}{https://github.com/cmusphinx/pocketsphinx}

\subsection{Since the Alpha}

\begin{itemize}
	\item Have fixed the initial crash of the app.
	\item Teragram is now able to be closed via speech command with the main menu 
	loaded.
	\item 2048 has been implemented with working speech commands. 
	\item Frozen Bubble has been implemented with working speech commands, however 
	some debugging still needs to happen with respect to the speed of the launcher 
	(see known issues for more).
	\item Have implemented the speech recognition on/off switch. It's currently 
	working with Teragram and the main menu.
	\item Have implemented visual feedback for the speech recognition, currently
	working within Teragram.
	\item In Teragram, have replaced random multiplication questions with times table
	practice that will adjust with the `too hard'/`too easy' buttons. 
      \item In Teragram, have implemented a button that when pressed
        will open up a powers of two quiz.
\end{itemize}

\pagebreak

\section{Intended improvements/Known Issues}

\subsubsection{Frozen Bubble Launcher Speed}
Frozen Bubble is an open source game that is controlled strictly by touch. 
In order for a user to be hands-free, we have had to manipulate the code 
to have the launcher work with respect to time. This currently is working
but not in a functional fashion as it travels far too fast. The issue is 
that the launcher moves with respect to int values, however, for it to be 
functional we need decimals values. When trying to manipulate this it 
throws a chain of bugs where the values are used elsewhere in the game. 

\subsubsection{More robust number entry}
The current way of inputing numbers using speech is finicky and prone to
errors. We are unsure whether or not there is some easier way to do this,
but we would like to try different methods.

\subsubsection{Speech Recognition off-switch across all games}
We have successfully implemented the on/off switch in the main menu and 
Teragram but still have to implement it into 2048 and FrozenBubble.

\subsubsection{Visual feedback across all games}
We have successfully implemented visual feedback in Teragram but still 
have to implement it into the main menu, 2048 and FrozenBubble.

\subsubsection{Help Screen}
Without this documentation, a user would have no way of knowing exactly 
what to say to the game to get it to do anything. Some kind of pop-up 
display you could call at any moment displaying the voice commands is
neccessary. 

\section{Testing}
Testing an application that is run by voice commands was not something
we felt could be done through automated methods without devoting a 
large amount of time that could instead be spent implementing features.
Instead, frequent testing was done manually either through emulators or 
by installing the apk on our own phones. For the Teragram game, we had 
extensive user testing performed by two children.
\pagebreak

\section{User Documentation}

\subsection{Teragram}
The game starts up at `level 1', and you can move to the easiest level
by tapping the `too hard' button or go up levels by tapping the `too
easy' button. Automatic leveling up happens when you get ten
questions right in a row, or levelling down when you get two wrong in
a row. The user is not aware of the actual level, it's just that
questions get easier or harder.

The game launches in addition mode, and the user can chose to move to
subtraction problems, or practice times tables or powers of two.

\subsection{2048}
2048 is a popular number name released on March 20, 2014. This game is
very simple; you can choose to slide the tiles up, down, left or right by
swipping the screen or saying `left', `right' `up' or `down'. Every
slide, all the digital squares will move to the direction of sliding
way. This is a really neat game, and there are many open source
implementations out there, but what makes it especially suitable for
our app is the simplicity of its controls.

\subsection{Frozen Bubble}
With colourful 3D rendered penguin animations and fun bubble sprites. 
This game is designed for the user to clear the screen by shooting the 
current bubble in the launcher at its respective coloured bubbles on 
the level. Saying `fire' or swipping up when they are wanting to shoot
the bubble.  

\subsection{Voice Controls}

You can control the game with the following voice commands:

\subsubsection{Main menu}

\begin{itemize}
  \item {\em\bf`game one'} - unsupported
  \item {\em\bf`Teragram' / 'game two'}  - this will take you to the game Teragram.
  \item {\em\bf`Twenty Fourty Eight' / 'game three'} - this will take you to the game 2048.
  \item {\em\bf`Frozen Bubble' / 'game four'} - this will take you to the game Frozen Bubble.
\end{itemize}

\subsubsection{Teragram}

\begin{itemize}
  \item {\em\bf`harder'} - Provides a harder question. 
  \item {\em\bf`easier'} - Provides an easier question.
  \item {\em\bf`new question'} - Provides a new question of the same type
  \item {\em\bf`addition'} -Provides an addition question
  \item {\em\bf`subtraction'} - Provides a subtraction question
  \item {\em\bf`multiplication'} - Provides a multiplication question 
  \item {\em\bf`number'} - Starts Number Input mode, see below
  \item {\em\bf`exit'} - Prompts a message confirming if they want to exit
 
\end{itemize}

\subsubsection{2048}

\begin{itemize}
	\item {\em\bf`left'} - All blocks will move to the left if possible. 
	\item {\em\bf`right'} - All blocks will move to the left if possible
	\item {\em\bf`up'} - All blocks will move to up if possible. 
	\item {\em\bf`down'} - All blocks will move to down if possible
	\item {\em\bf`exit'} - Prompts a message confirming if they want to exit
	
\end{itemize}

\subsubsection{Frozen Bubble}

\begin{itemize}
	\item {\em\bf`fire' or `now'} - Will fire the bubble
	\item {\em\bf`exit'} - Prompts a message confirming if they want to exit
	
\end{itemize}

\pagebreak

\subsubsection{Number Input}
\begin{itemize}
  \item {\em\bf numbers 0 to 9 inclusive} – Appends the spoken number
    to the end of your answer, ie. Speaking `one', `two' will give you
    12. Note that 0 is to be spoken `zero'
  \item {\em\bf `okay'} - Submits the current answer and exits Number
    Input mode
  \item {\em\bf `back'} - Removes the last number from your answer
  \item {\em\bf `clear'} - Removes the entirety of your answer
  \item {\em\bf `cancel'} - Exits Number Input mode
\end{itemize}

When entering numbers, please wait until the number you have just spoken appears
in the box before trying to input the next one.
Good Luck!

\section{What we stole}

The code for the speech recognizer setup/permissions/asset copying is
taken almost directly from the pocketSphinx demo application here:
\url{https://github.com/cmusphinx/pocketsphinx-android-demo}

The phoneme definitions in our nktd.dic file are selectively taken
from the cmudict-en-us.dict file in that same demo.

The acoustic model we use is provided by CMUSphinx at
\href{https://sourceforge.net/projects/cmusphinx/files/Acoustic\%20and\%20Language\%20Models/US\%20English/}{sourceforge.net}

The game FrozenBubble taken from
\url{https://github.com/kthakore/frozen-bubble}

The game 2048 taken from
\url{https://github.com/BuddyBuild/2048-Android}
    


\end{document}
