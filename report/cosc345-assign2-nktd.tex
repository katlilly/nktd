\documentclass[11pt, oneside]{article}  
\usepackage{geometry}                	
\geometry{a4paper}
\usepackage[parfill]{parskip}		
\usepackage{graphicx}	
\usepackage{pgfgantt}
\usepackage{pgfcalendar}	
\usepackage{pgfkeys}	
\usepackage{blindtext}					
\usepackage{amssymb}
\usepackage{lscape}
\usepackage{hyperref}
\hypersetup{
    colorlinks=true,
    linkcolor=blue,
    filecolor=magenta,      
    urlcolor=cyan,
}
\urlstyle{same} 

\title{COSC 345 Assignment 2} \author{Nathan Kennedy \texttt{1006384},
  Kat Lilly \texttt{8867318}, Tobyn Packer \texttt{5003352}, and
  \\Daniel Davidson \texttt{375621}} \date{\today}

\begin{document}

\maketitle

\section*{Alpha Release}

At this point we have an android application consisting of a stylised
main menu and from which a user may chose a game to play. The only
game currently working is `Teragram', a maths game for young
children. Choice of which game to play, as well as play of the math
game can be done both by tapping the buttons on screen, or by
voice command only.

The application uses pocketSphinx to interpret speech commands.

\subsection*{Known issues}

Currently the user can't exit from Teragram via speech commands. Also,
after manually exiting from Teragram the speech recognition does not
restart.

\pagebreak
\section*{User Documentation}

Upon startup, you will be greeted with the main menu. From there:

\subsection*{For the armed}

You can start the maths game `Teragram' from the main menu by tapping
the teragram button. The game starts up at `level 1', and you can move
to the easiest level, zero, by tapping the 'too hard' button or go up
levels by tapping the `too easy' button. Automatic levelling up
happens when you get ten questions right in a row, or levelling down
when you get two wrong in a row. You can 


\subsection*{For the unarmed}

You can control the game with the following voice commands:

\subsubsection*{Main menu}

\begin{itemize}
  \item {\em\bf`game one'} - unsupported
  \item {\em\bf`game two'}  - this will take you to the game Teragram.
  \item {\em\bf`game three'} - unsupported
  \item {\em\bf`game four'} - unsupported
\end{itemize}

\subsubsection*{Teragram}

\begin{itemize}
  \item {\em\bf`harder'} - Provides a harder question. 
  \item {\em\bf`new question'} - Provides a new question of the same type
  \item {\em\bf`addition'} -Provides an addition question
  \item {\em\bf`subtraction'} - Provides a subtraction question
  \item {\em\bf`multiplication'} - Provides a multiplication question 
  \item {\em\bf`number'} - Starts Number Input mode, see below
  \item {\em\bf`exit'} - unsupported
\end{itemize}

\subsubsection*{Number Input}
\begin{itemize}
  \item {\em\bf numbers 0 – 9 inclusive} – Appends the spoken number
    to the end of your answer, ie. Speaking `one', `two' will give you
    12. Note that 0 is to be spoken `zero'
  \item {\em\bf `enter'} - Submits the current answer and exits Number
    Input mode
  \item {\em\bf `back'} - Removes the last number from your answer
  \item {\em\bf `clear'} - Removes the entirety of your answer
  \item {\em\bf `cancel'} - Exits Number Input mode
\end{itemize}
    
\pagebreak
\section*{What we learned}

\subsection*{Android is not `just Java'}

When we first began writing code for our project, we began by writing
a Java application. It wasn't until we had gotten a not insignificant
way into the development process that we learned that android uses
different graphical libraries and not long after, that it used
different audio libraries too. Everything we had written up to that
point needed to be translated into android-friendly code or scrapped
entirely.

\subsection*{Testing is hard}

Our initial efforts were focused on getting an integrated testing
environment up and running. This involved setting up travis CI and
writing tests to ensure nothing was being broken between pushes. As it
turns out, setting up a travis environment for android is not
simple. There are multiple versions of the Android SDK to worry about
and licence agreements and of course, gradle. We did ultimately get a
travis environment set up in the end, so we do know when something is
pushed that won't build.

\subsection*{Android Studio is hard}

All of us were unfamiliar with the Android Studio IDE and all of us
have struggled with it. Some of us were having trouble getting it to
acknowledge that the Android SDK was in fact installed on our
computers, while for some of us, code that built fine on somebody’s
computer would not on anothers. It seems to be a problem that occurs
whenever the project’s file structure was changed. Sometimes
restarting Android Studio fixes the problem, sometimes explicitly
forcing Android Studio to sync the project with Gradle fixes the
problem and sometimes it won’t. In the end we had two out of four
computers building the same code, enabling us to get something done.

\subsection*{Adapting other people's projects is hard}

When things were starting to look bleak in terms of how much we would
be able to present for this deliverable, we tried to adapt open-source
android games and impose our speech commands onto them. Coupled with
our Android Studio file structure problems, this was a disaster.
Manifests needed to be edited, other people’s code needed to be
learnt, and many bugs needed to be found and fixed.  We did waste
plenty of time on this idea, before writing a new game from scratch
that has fewer bugs in it and is more well suited to voice control.


\section*{What we stole}

The code for the speech recognizer setup/permissions/asset copying is
taken almost directly from the pocketSphinx demo application here:
\url{https://github.com/cmusphinx/pocketsphinx-android-demo}

The phoneme definitions in our nktd.dic file are selectively taken
from the cmudict-en-us.dict file in that same demo.

The acoustic model we use is provided by CMUSphinx at
\href{https://sourceforge.net/projects/cmusphinx/files/Acoustic\%20and\%20Language\%20Models/US\%20English/}{sorceforge.net}


\end{document}


