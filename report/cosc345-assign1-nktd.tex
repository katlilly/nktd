\documentclass[11pt, oneside]{article}  
\usepackage{geometry}                	
\geometry{a4paper}                   			
\usepackage[parfill]{parskip}		
\usepackage{graphicx}	
\usepackage{pgfgantt}
\usepackage{pgfcalendar}	
\usepackage{pgfkeys}	
\usepackage{blindtext}					
\usepackage{amssymb}

\title{COSC 345 Assignment 1} \author{Nathan Kennedy \texttt{1234567},
  Kat Lilly \texttt{8867318}, Tobyn Packer \texttt{1234567}, and
  \\Daniel Davidson \texttt{1234567}} \date{\today}

\begin{document}
\maketitle

\section*{What we are going to build}

We intend to build an app containing a suite of games that can be
played without the use of one's hands. Entry to the app as well as
control of the games will all be voice controlled.

We have begun with a tetris game as a way to get started and learn the
various things we'll need to learn, such as how to use open source
speech recognition software, how to write an android app, etc. Tetris
is a game that has very simple keyboard inputs, and for which there
are many online tutorials for writing the game, so there shouldn't be
significant challenges beyond capturing the voice commands.

After this we intend to add additional games. We intend to explore the
possibility of games that may be more interesting to control by voice
than tetris is. One can imagine games where the gameplay could be
changed or improved by controlling this with changes in the pitch or
loudness of your voice rather than keyboard or touch screen inputs,
but don't have any well formed ideas about this yet.

The working title of our app is `nktd', the most pleasing permutation
of our initials.


\subsection*{Intended users}

Our target users are anyone who does not have full use of their hands,
either permanently or temporarily. The app will be launched with the
voice control in the operating system, and the user will be able to
select a game and play it using only voice commands. 

Our intended users are defined not by a particular illness, injury
or other source of disability, but by this particular aspect of their
interaction with a device, that they are unable to use their hands at
this time. Our users may include amputees, people with a hand tremor,
people with a broken arm, and so on.

We are beginning with voice control as a substitute for key presses
and touch screen control, but also want to explore the possibility of
using eye tracking for some aspects of our app. With our current
understanding, it seems that this may be a bigger, harder problem
than using voice control, and is not one of our primary goals.


\subsection*{Platforms}

We will build an Android app that runs on both smartphones and
tablets. If one or more of the games we write is suitable for a very
small screen we will also consider including a smartwatch version.

We intend to have an optional login function in our app, that would
allow the user's saved settings and high scores to be transferred between
devices.


\section*{How we will build it}

The game logic and the Android app will be written in Java. We are
using open source voice recognition software *name* from *github link*

{\em This section and team member roles both need more detail}

\subsection*{Team member roles}

\begin{description}
\item [Nathan] Android app development, user interface and graphics
\item [Kat] Reporting, documentation, writing game logic
\item [Tobyn] Graphics, ideas for new games
\item [Daniel] Voice recognition software
\end{description}


\subsection*{Managing the workflow}

We will manage our project using the Scrum framework, currently
working with two week sprints. We have a regular team meeting on
Mondays (more often when needed), where we catch each other up on what
we have acheived over the previous week and what needs doing next. So
far these meetings have mostly been about evolving our idea of what we
are actually trying to make.

We have a Slack workspace for our project that is linked to the nktd
repository on github, which we use to communicate about the project
between meetings.

We also are using Google Drive to host a spreadsheet with our Scrum
workflow stuff.

We are using the github wiki to record what was discussed at meetings
and for notes about how aspects of the project work.


\section*{Project Timeline}

{\em Kat currently working on a couple of gantt charts to put here}

\begin{ganttchart}[
    hgrid,
    vgrid,
    time slot format=isodate-yearmonth,
    time slot unit=month
  ]{2012-03}{2014-1}
\gantttitlecalendar{year, month} \\
  \ganttbar{}{2012-05}{2013-01}
\end{ganttchart}

%\begin{ganttchart}[
%hgrid,
%vgrid,
%x unit=4mm,
%time slot format=isodate-yearmonth,
%time slot unit=month
%]{2018-04-16}{2018-05-28}
%\gantttitlecalendar{year, month=name} \\
%%\gantttitlecalendar{year, month, day, week=3, weekday} \\
%%\ganttbar{}{2018-01-14}{2013-01-17}
%\end{ganttchart}
%
%\begin{ganttchart}{3}{4}[vgrid]{2259.55}{2259.67}
%  \gantttitle{2018}
%  \gantttitlecalendar{year, month=shortname, day} \\
%\end{ganttchart}

%\begin{ganttchart}[
%    hgrid,
%    vgrid,
%    time slot format=isodate-yearmonth,
%    time slot unit=month
%  ]{2012-03}{2014-1}
%\gantttitlecalendar{year, month} \\
%  \ganttbar{}{2012-05}{2013-01}
%\end{ganttchart}

\begin{ganttchart}{1}{12}
  \gantttitle{2011}{12} \\
  \gantttitlelist{1,...,12}{1} \\
\ganttgroup{Group 1}{1}{7} \\
\ganttbar{Task 1}{1}{2} \\
\ganttlinkedbar{Task 2}{3}{7} \ganttnewline
\ganttmilestone{Milestone}{7} \ganttnewline
\ganttbar{Final Task}{8}{12}
\ganttlink{elem2}{elem3}
\ganttlink{elem3}{elem4}
\end{ganttchart}

%
%\begin{ganttchart}{1}{12}
%\gantttitle{2018}{7} \\
%\gantttitlelist{1,...,7}{1} \\
%\ganttgroup[progress=50]{Group 1}{1}{7} \\
%\ganttbar{Task 1}{1}{2} \\
%\ganttbar{Task 1}\\
%\ganttlinkedbar[progress=20]{Task 2}{3}{7} \ganttnewline
%\ganttmilestone{Assignment 1 due}{7} \ganttnewline
%\ganttbar{Final Task}{8}{12} \ganttnewline
%\ganttmilestone{Alpha release}{7} \ganttnewline
%\ganttlink{elem2}{elem3}
%\ganttlink{elem3}{elem4}
%\ganttmilestone{Beta release}{6} \ganttnewline
%\ganttlink{elem1}{elem3}
%\end{ganttchart}
%




\end{document}




%\begin{ganttchart}{1}{12}
%\gantttitle{2018}{12} \\
%\gantttitlelist{1,...,12}{1} \\
%\ganttgroup[progress=50]{Group 1}{1}{7} \\
%\ganttbar{Task 1}{1}{2} \\
%\ganttbar{Task 1}\\
%\ganttlinkedbar[progress=20]{Task 2}{3}{7} \ganttnewline
%\ganttmilestone{Assignment 1 (report)}{7} \ganttnewline
%\ganttbar{Final Task}{8}{12} \ganttnewline
%\ganttmilestone{Assignment 2 (alpha release)}{7} \ganttnewline
%\ganttlink{elem2}{elem3}
%\ganttlink{elem3}{elem4}
%\ganttlink{elem1}{elem3}
%\end{ganttchart}




\end{document}  

