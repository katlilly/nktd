\documentclass[11pt, oneside]{article}  
\usepackage{geometry}                	
\geometry{a4paper}
\usepackage[parfill]{parskip}		
\usepackage{graphicx}	
\usepackage{pgfgantt}
\usepackage{pgfcalendar}	
\usepackage{pgfkeys}	
\usepackage{blindtext}					
\usepackage{amssymb}
\usepackage{hyperref}
\usepackage{lscape}
\newcounter{myWeekNum}
\stepcounter{myWeekNum}

\newcommand{\myWeek}{\themyWeekNum
    \stepcounter{myWeekNum}
    \ifnum\themyWeekNum=53
         \setcounter{myWeekNum}{1}
    \else\fi
}

\title{COSC 345 Assignment 1} \author{Nathan Kennedy \texttt{1006384},
  Kat Lilly \texttt{8867318}, Tobyn Packer \texttt{5003352}, and
  \\Daniel Davidson \texttt{375621}} \date{\today}

\begin{document}

\maketitle

\section*{What we are going to build}

We intend to build an app containing a suite of games that can be
played without the use of one's hands. Entry to the app as well as
control of the games will be by voice command.

We have begun with a tetris game as a way to get started, to
familiarize ourselves with speech recognition software, and to learn
how to write an Android app. Tetris is a game that has very simple
keyboard inputs, and for which there are many online tutorials for
writing the game, so there shouldn't be significant challenges beyond
capturing the voice commands.

After this we intend to add at least one additional game. We are
exploring the possibility of games that may be more interesting to
control by voice than tetris is, where the gameplay is improved or
changed by being voice controlled. For example, there is a Mario-style
game called ``Scream Go''
\url{https://www.youtube.com/watch?v=IyvjpgS9TAs}, where a character
navigates a world with walking and jumping movements, which are
controlled by pitch and loudness of the player's voice. Another
example is ``Full Voice Throttle''
\url{https://kushulain.itch.io/full-voice-throttle}, a motorbike
racing game where acceleration is achieved by increasing the pitch of
the player's voice, and the gear is changed by starting again at a
lower pitch.

The working title of our app is `nktd', the most pleasing permutation
of our initials. \url{http://github.com/katlilly/nktd}


\subsection*{Intended users}

We spoke to the manager of the disability support unit here at Otago,
and talked with her about what the real needs of students and staff
here are. Her main point was that mental health issues are the biggest
issue, and so we briefly explored the possibility of a project aimed
at supporting Otago students in this area. However, none of us were
particularly keen to take on such a weighty issue as this, and we also
felt that a project like this would involve a lot of work writing
content and doing research about non CS topics, which is not how we
would prefer to spend most of our effort.

Instead we have chosen to write games that can be controlled without
the use of hands. None of us had written games before, and we were all
keen for that particular challenge. In addition, we believe this will
be a project where there is a very achievable minimum product that we
can get started with, as well as lots of opportunity to expand on it
within the timescale of one academic year. We feel that this will be a
good project for learning a lot of the skills we want to get out of
this software engineering course.

Our target users are anyone who does not have full use of their hands,
either permanently or temporarily. The app will be launched with the
voice control in the operating system, and the user will be able to
select a game and play it using only voice commands. 

Our intended users are defined not by a particular illness, injury or
other source of disability, but are unable to use their hands, and we
are focusing on this particular interaction with a device. Our users
may include amputees, quadriplegics, or people with a hand tremor, a
broken arm, arthritis, and so on.

Of course our users will not be restricted to those who are unable to use 
their hands, and we hope that our games will be fun to play for anyone 
without an impairment.

A particular user might be a child who has broken both their wrists
and is in pain. We know from personal experience that playing games
is not only a solution to boredom but can be really good for coping
with pain.

We assume that our users have already solved the problem of holding
and unlocking their device, and are using the operating system's voice
recognition features. We will use the system-provided Voice Actions to
allow users to open our app by a voice command such as ``OK Google,
play tetris''.

No specialised hardware will be necessary, just the built in
microphone that all phones and tablets have.

%look up best practice for writing software for this disability....?

We are beginning with voice control as a substitute for key presses
and touch screen control, but also want to explore the possibility of
using eye tracking for some aspects of our app. It is likely that this
will be more than we can get working well within the timeframe of this
project, but we will make the call after further research, and not at
the expense of getting voice control working well.


\subsection*{Platforms}

We will build an Android app that runs on both smartphones and
tablets. If one or more of the games we write is suitable for a very
small screen we will also consider including a smartwatch version.

We intend to have an optional login function in our app, that would
allow the user's saved settings and high scores to be transferred between
devices.


\section*{How we will build it}

The game logic and the Android app will be written in Java. We are
using open source voice recognition software CMUSphinx from
\url{https://cmusphinx.github.io}. More specifically, we are using sphinx4.

The speech recognition system requires three parts:
\begin{itemize}
\item An acoustic model consisting of acoustic properties for each
  `phone' or unit of sound.
\item A dictionary mapping combinations of phones to words.
\item A grammar model that tells the system valid combinations of
  words it will hear.
\end{itemize}

The dictionary and the acoustic model can be shared among all games we
make, but we will need to write a new grammar model for each game.

We will be using one of CMUSphinx's pre-built acoustic models
downloaded from \url{https://sourceforge.net/projects/cmusphinx}. Our
dictionaries will be generated for us by CMUSphinx's LMTool here:
\url{www.speech.cs.cmu.edu/tools/lmtool-new.html}. The grammar models
we need to write ourselves in JSFG - Java Speech Grammar Format

The speech recognition will only work accurately on audio of the same
sample rate/number of channels as the audio its acoustic model was
trained on. The CMUSphinx model was trained on 16Khz, 16bit Mono
audio. This means that potentially we will need to downsample the
audio being received from the user's microphone before it is
interpreted. We can do this using the Java Sound API.

We will also need to learn how to incorporate pitch and volume
information into our game control, this won't be done with CMUSphinx,
and we have not yet done any real research into this aspect of the
project.


\subsection*{Team member roles}

\begin{description}
\item [Nathan] Android app development, designing user interface and
  graphics.
\item [Kat] Writing reports and meeting minutes, writing game logic,
  some testing.
\item [Tobyn] Graphics, ideas for new games and coding them. Writing
  user documentation
\item [Daniel] Lead test developer, working with open source voice
  recognition software, speech recognition and audio inputs.
\end{description}


\subsection*{Managing the workflow}

We will manage our project using the Scrum framework, currently
working with two week sprints. We have a regular team meeting on
Mondays (more often when needed), where we catch each other up on what
we have achieved over the previous week and what needs doing next. So
far these meetings have mostly been about evolving our idea of what we
are actually trying to make.

We have a Slack workspace for our project that is linked to the nktd
repository on github, which we use to communicate about the project
between meetings.

We also are using Google Drive to host a spreadsheet with our Scrum
workflow.

We are using the github wiki to record what was discussed at meetings
and for notes about how aspects of the project work.

We intend to use a continuous integration tool to keep our code in a non-broken 
state as much as possible, but don't yet have this functioning.

\section*{Project Timeline}

Our alpha release (Assignment 2) will be an app in which the user can
play tetris purely with voice commands, and will work on both Android
phones and Android tablets.

Improvements in our beta release (Assignment 3) will include the
addition of a second game, and the ability to log in to the app and
keep your preferences and high scores between devices.

Our finished product (Assignment 4) will be a thoroughly tested, less
buggy version than Assignment 3, with faster and more accurate
interpretation of voice commands as well as the possible addition of a
third game and/or a smartwatch version.

See Figure \ref{fig:ganttsem1} for our Semester One timeline, and
Figure \ref{fig:ganttsem2} for a broad brush timeline for Semester Two.


% first semester gantt chart
\begin{figure}
\begin{ganttchart}[
        title/.append style={fill=black!10},
        hgrid,
        %vgrid={draw=none, dotted},
        x unit=2.3pt,
        y unit title=0.6cm,
        y unit chart=0.6cm,
        title label font={\small},
        time slot format=simple,
        milestone/.append style={xscale=4} % adjusting for my small x unit
        ]{1}{84} %215 days between March 1st and October 1st
 
 % write year, month, and sprint headers
    \gantttitle{2018}{84} \\
    \gantttitle{March}{26} % set start date to 5th march
    \gantttitle{April}{30}
    \gantttitle{May}{28}\\ 

    \gantttitle{1}{14} % there are 15 two week sprints in the project timeline
    \gantttitle{2}{14}
    \gantttitle{3}{14}
    \gantttitle{4}{14}
    \gantttitle{5}{14}
    \gantttitle{6}{14}\\
% end of header stuff

{\small % set text size for tasks to small

% Stage 1
\ganttbar{\small Develop idea}{1}{14}\\
\ganttbar{\small Research voice recognition}{15}{28} \\
\ganttbar{\small Tetris game logic}{15}{28} \\
\ganttbar{\small Create basic app}{15}{28} \\
\ganttbar{\small Tetris-specific voice commands}{29}{42} \\
\ganttbar[progress=20]{\small Tetris graphics in java}{29}{56} \\
\ganttbar[progress=50]{\small Researching other games}{29}{56} \\
\ganttmilestone{1st Report due}{47} \\

% Stage 2 - to Alpha Release
\ganttbar[bar/.append style={fill=gray}]{\small App works on tablet and phone}{57}{70} \\
\ganttbar[bar/.append style={fill=gray}]{\small Start writing a second game}{57}{84} \\
\ganttbar[bar/.append style={fill=gray}]{\small Tetris running in Android app}{71}{81} \\
\ganttbar[bar/.append style={fill=gray}]{\small Tetris controllable by voice}{71}{81} \\
\ganttmilestone{Alpha Release}{84}

} % end set font size

% link dependent elements
%\ganttlink{elem0}{elem5}
%\ganttlink{elem0}{elem1}

\end{ganttchart}
\caption{Semester One timeline. We have divided our first semester efforts into 6 two-week sprints, beginning March 6th and ending with Assignment Two deadline on Monday 28th May.}
\label{fig:ganttsem1}
\end{figure}


% second semester gantt chart
\begin{figure}
\begin{ganttchart}[
        title/.append style={fill=black!10},
        hgrid,
        x unit=2.3pt,
        y unit title=0.6cm,
        y unit chart=0.6cm,
        title label font={\small},
        time slot format=simple,
        milestone/.append style={xscale=4} % adjusting for my small x unit
        ]{1}{84} %215 days between March 1st and October 1st
 
 % write year, month, and sprint headers
    \gantttitle{2018}{84} \\
    \gantttitle{July}{22}     % semester 2 starts 9th July
    \gantttitle{August}{31}
    \gantttitle{September}{30}
    \gantttitle{}{1}\\ % estimated date of final assignment is October 1st

    \gantttitle{7}{14} 
    \gantttitle{8}{14}
    \gantttitle{9}{14}
    \gantttitle{10}{14}
    \gantttitle{11}{14}
    \gantttitle{12}{14}\\
% end of header stuff

{\small % set text size for tasks to small

% Stage 3 - to Beta Release
\ganttbar[bar/.append style={fill=gray}]{\small User testing of alpha release}{1}{14} \\
\ganttbar[bar/.append style={fill=gray}]{\small Complete logic for second game}{1}{14} \\
\ganttbar[bar/.append style={fill=gray}]{\small Voice commands for second game}{15}{28} \\
\ganttbar[bar/.append style={fill=gray}]{\small Add login feature}{15}{28} \\
\ganttbar[bar/.append style={fill=gray}]{\small Possible smartwatch version?}{29}{42} \\
\ganttmilestone{Beta Release}{42}\\

% Stage 4 - To finished product
\ganttbar[bar/.append style={fill=gray}]{\small User testing of beta release}{43}{56} \\
\ganttbar[bar/.append style={fill=gray}]{\small Addition of 3rd game?}{43}{56} \\
\ganttbar[bar/.append style={fill=gray}]{\small Fast interpretation of commands}{57}{70} \\
\ganttbar[bar/.append style={fill=gray}]{\small 90\% voice command accuracy}{57}{70} \\
\ganttbar[bar/.append style={fill=gray}]{\small Don't panic}{71}{83} \\
\ganttmilestone{Finished product}{84}

} % end set font size

% link dependent elements
%\ganttlink{elem2}{elem6}
%\ganttlink{elem3}{elem6}
%\ganttlink{elem6}{elem8}

\end{ganttchart}
\caption{Rough outline of our Semester Two timeline, composed of 6 two-week sprints beginning Monday 9th July, with an estimated due date for the final product of Monday 1st October.}
\label{fig:ganttsem2}
\end{figure}


\end{document}





